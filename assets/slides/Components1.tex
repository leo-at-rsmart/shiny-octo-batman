\documentclass[xcolor=dvipsnames,14pt,professionalfonts]{beamer}
\usepackage{minted}
\usepackage{etoolbox}
\usepackage[T1]{fontenc}
\usepackage{lmodern}
\usepackage[no-math]{fontspec} 
\usetheme{rsmarttraining}
\usefonttheme{professionalfonts}
\usecolortheme{dolphin}

%\definecolor{foreground}{gray}{0}
%\definecolor{background}{gray}{1}
%\definecolor{keyword}{rgb}{0.2,0.2,0.8}
%\definecolor{warning}{rgb}{0.8,0.2,0.2}
%\definecolor{shadow}{gray}{0.35}
%\definecolor{hide}{gray}{0.9}
%\definecolor{figure}{rgb}{1,0.7,0}
%\definecolor{title}{rgb}{25,240,250}
\definecolor{title}{rgb}{0.09,0.30,0.38}
\definecolor{frametitle}{rgb}{1,1,1}
\definecolor{normal}{rgb}{0,0,0}

%\usecolortheme[named=keyword]{structure}

\setbeamercolor{title}{fg=title}
\setbeamercolor{frametitle}{fg=frametitle}
\setbeamercolor{section in toc}{fg=foreground}
\setbeamercolor{normal text}{bg=brown!46,fg=normal}

\setbeamerfont{structure}{family=\fontspec{Georgia},series=\bfseries} 
\setbeamerfont{subtitle}{family=\fontspec{Helvetica},series=\bfseries} 
\begin{document}

\title{KRAD Training}
\subtitle{Components: Part 1}
\author[Leo]{Leo Przybylski}

\usebackgroundtemplate%
{%
    \includegraphics[width=\paperwidth,height=\paperheight]{../img/header.png}%
}

{
\usebackgroundtemplate{\includegraphics[width=\paperwidth]{../img/title.png}}%
\begin{frame}[plain]
  \titlepage
\end{frame}
}

\begin{frame}{Overview}
  \begin{itemize}
  \item What is a Component?
  \item What Components are Composed of
  \item Freemarker Templates
  \item Rice customizations of Freemarker
  \item Cases for Custom Components
  \end{itemize}
\end{frame}

\begin{frame}{What is a Component?}
  \begin{itemize}
    \item Components are any UI element rendered with the UIF.
    \item Examples are:
      \begin{itemize}
        \item Elements
        \item Fields
        \item Containers
        \item Layout Managers
        \item Views
      \end{itemize}
  \end{itemize}
\end{frame}

\begin{frame}{What is a Component?}
\end{frame}

\begin{frame}{Components are Composed of}
  \begin{itemize}
        \item Component Class
        \item Freemarker Template
        \item DataDictionary definition
  \end{itemize}
\end{frame}

\begin{frame}{Freemarker Templates}
  \begin{itemize}
    \item Generally, not necessary to modify or create
    \item Templates used for components
    \item Can take input from the Component Class
    \item Additional input is also available through template.options.
  \end{itemize}
\end{frame}

\begin{frame}{Rice Freemarker Customizations}
  \begin{itemize}
  \item Rice made customizations to Freemarker in a forked version of
    it.
  \item Among these customizations is the ability to make inline calls
    to templates.
  \item Inline templates must be registered with the
    \texttt{InlineTemplateElement.registerAdaptor}
    method. This is done on Spring initialization with
    \texttt{FreeMarkerInlineRenderBootstrap}
  \item Adding your own is as simple as adding a similar class to your
    module.
  \item \texttt{InlineTemplateElement.registerAdaptor} registers a
    class that implements the \texttt{InlineTemplateAdaptor} interface
  \end{itemize}
\end{frame}

\begin{frame}{Rice Freemarker Customizations}
  \begin{itemize}
    \item \texttt{InlineTemplateAdaptor} is responsible for actually
      rendering your Inline template.
  \end{itemize}
\end{frame}

\begin{frame}{Cases for Custom Components}
  \begin{itemize}
    \item If you need a component that doesn't exist in KRAD, you may
      need to create a custom one. Some examples are:
      \begin{itemize}
        \item Spinner
        \item Coverflow
      \end{itemize}
    \item If an existing component isn't giving the features or
      flexibility needed. One example of this is layout
      managers. Collections only have two which are TableLayout and
      StackedLayout. If you need a different layout, you need to make
      your own.
  \end{itemize}
\end{frame}

\end{document}
