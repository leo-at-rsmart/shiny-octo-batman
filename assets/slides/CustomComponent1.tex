\documentclass[xcolor=dvipsnames,14pt,professionalfonts]{beamer}
\usepackage{minted}
\usepackage{etoolbox}
\usepackage[T1]{fontenc}
\usepackage{lmodern}
\usepackage[no-math]{fontspec} 
\usetheme{rsmarttraining}
\usefonttheme{professionalfonts}
\usecolortheme{dolphin}

%\definecolor{foreground}{gray}{0}
%\definecolor{background}{gray}{1}
%\definecolor{keyword}{rgb}{0.2,0.2,0.8}
%\definecolor{warning}{rgb}{0.8,0.2,0.2}
%\definecolor{shadow}{gray}{0.35}
%\definecolor{hide}{gray}{0.9}
%\definecolor{figure}{rgb}{1,0.7,0}
%\definecolor{title}{rgb}{25,240,250}
\definecolor{title}{rgb}{0.09,0.30,0.38}
\definecolor{frametitle}{rgb}{1,1,1}
\definecolor{normal}{rgb}{0,0,0}

%\usecolortheme[named=keyword]{structure}

\setbeamercolor{title}{fg=title}
\setbeamercolor{frametitle}{fg=frametitle}
\setbeamercolor{section in toc}{fg=foreground}
\setbeamercolor{normal text}{bg=brown!46,fg=normal}

\setbeamerfont{structure}{family=\fontspec{Georgia},series=\bfseries} 
\setbeamerfont{subtitle}{family=\fontspec{Helvetica},series=\bfseries} 
\begin{document}

\title{KRAD Training}
\subtitle{Exercise: Custom Components}
\author[Leo]{Leo Przybylski}

\usebackgroundtemplate%
{%
    \includegraphics[width=\paperwidth,height=\paperheight]{../img/header.png}%
}

{
\usebackgroundtemplate{\includegraphics[width=\paperwidth]{../img/title.png}}%
\begin{frame}[plain]
  \titlepage
\end{frame}
}

\begin{frame}{Goals}
  \begin{itemize}
    \item Learn about Framemaker templating.
    \item Create a component that uses template options.
    \item Understand how to update/modify existing components.
  \end{itemize}
\end{frame}

\begin{frame}{Instructions}
  \begin{enumerate}
    \item checkout \textbf{exercise-krad-custom-component}
    \item Create a \texttt{FreeMarkerCoverFlowAdapter} in the
      \texttt{trnapp.bookstore} package
    \item It will look like the following slide
  \end{enumerate}
\end{frame}

 \begin{frame}[fragile]{FreeMarkerCoverFlowAdapter}
   \begin{minted}[fontsize=\scriptsize]{java}
public class FreeMarkerCoverFlowAdapter 
         implements InlineTemplateAdaptor, 
                            Serializable {
    @Override
    public void accept(Environment env) 
            throws TemplateException, IOException {
        final CollectionGroup group = 
                FreeMarkerInlineRenderUtils
                        .resolve(env, "group", CollectionGroup.class);

        final Macro fmMacro = 
                (Macro) env.getMainNamespace().get("coverFlow");
        final Map<String, Object> args = 
                new java.util.HashMap<String, Object>();
        args.put("group", group);
        InlineTemplateUtils.invokeMacro(env, fmMacro, args, null); 
   }
}
    \end{minted}
\end{frame}

\begin{frame}{Instructions}
  \begin{enumerate}
    \item A bootstrapping class is now required to
      register the adapter with Spring
    \item Create a new class \texttt{FreeMarkerInlineRenderBootstrap}
      in the \texttt{trnapp.bookstore} package.
    \item Add the code from the following slide
  \end{enumerate}
\end{frame}

 \begin{frame}[fragile]{FreeMarkerInlineRenderBootstrap}
   \begin{minted}[fontsize=\scriptsize]{java}
public class FreeMarkerInlineRenderBootstrap 
        implements InitializingBean {
    @Override
    public void afterPropertiesSet() throws Exception {
        InlineTemplateElement.registerAdaptor("coverFlow", 
                new FreeMarkerCoverFlowAdapter());
    }
}
    \end{minted}
\end{frame}

\begin{frame}{Instructions}
  \begin{enumerate}
    \item Now the \texttt{FreeMarkerInlineRenderBootstrap} needs to be
      added to the Spring configuration in order to register the
      adapter on startup.
    \item Add the following to the 
      \texttt{src/main/resources/trnapp/bookstore/Module.xml}
  \end{enumerate}
\end{frame}

 \begin{frame}[fragile]{Module.xml}
   \begin{minted}[fontsize=\scriptsize]{xml}
<bean class="trnapp.bookstore.FreeMarkerInlineRenderBootstrap" />
    \end{minted}
\end{frame}

\begin{frame}{Instructions}
  \begin{enumerate}
    \item Now we need to modify the \textbf{resultsGroup} of the
      \texttt{Book-LookupView} to point to a different template.
    \item Open the
      \texttt{src/main/resources/trnapp/bookstore/Book.xml}
    \item Add the following to the \texttt{Book-LookupView}. This will
      modify the layout manager to use a coverflow layout when
      displaying lookup results.
  \end{enumerate}
\end{frame}

 \begin{frame}[fragile]{Book.xml}
   \begin{minted}[fontsize=\scriptsize]{xml}
<bean id="Book-LookupView" parent="Uif-LookupView">
...
...
	<property name="resultsGroup">
    	<bean parent="Uif-LookupResultsCollectionSection">
    		<property name="layoutManager">
    			<bean parent="Uif-TableCollectionLayout" 
    				p:template="/WEB-INF/ftl/coverflow.ftl" 
    				p:templateName="coverFlow" />
    		</property>
    	</bean>
    </property>
...
...
</bean>
    \end{minted}
\end{frame}

 \begin{frame}[fragile]{coverflow.ftl}
   This is some background about coverflow.ftl. Since it has been
   provided, we don't need to actually do anything here.
  \begin{enumerate}
    \item This is the file that is going to handle the layout of
      collection items resulting from the Book lookup.
    \item The following is the definition of the template. In
      Freemarker, it's called a \textbf{macro}
   \begin{minted}[fontsize=\scriptsize]{xml}
     <#macro coverFlow items manager container>
   \end{minted}
 \end{enumerate}
\end{frame}

 \begin{frame}[fragile]{coverflow.ftl}
  \begin{enumerate}
   \item The \texttt{items} list is iterated to display the values.
   \begin{minted}[fontsize=\scriptsize]{xml}
<#list items as item>
  <#if item_index == 0>
    <li data-target="#lookup-results-coverflow" 
        data-slide-to="${item_index + 1}" class="active"></li>
  <#else>
    <li data-target="#lookup-results-coverflow" 
        data-slide-to="${item_index + 1}"></li>
  </#if>
</#list>
   \end{minted} 
   \item \texttt{item\_index} is the current index within the
     loop. This can be used to determine if we're on the first or last
     item. In this case, the first item is always active.
 \end{enumerate}
\end{frame}

\begin{frame}{That's it}
  \begin{enumerate}
    \item Explore ways to improve on this. Display values from the
      items like author and title.
    \item Find a way to add url's to the results to return a value or
      display an inquiry.
  \end{enumerate}
\end{frame}

\end{document}
