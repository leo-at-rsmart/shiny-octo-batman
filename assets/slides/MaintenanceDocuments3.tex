\documentclass[xcolor=dvipsnames,14pt,professionalfonts]{beamer}
\usepackage{minted}
\usepackage{etoolbox}
\usepackage[T1]{fontenc}
\usepackage{lmodern}
\usepackage[no-math]{fontspec} 
\usetheme{rsmarttraining}
\usefonttheme{professionalfonts}
\usecolortheme{dolphin}

%\definecolor{foreground}{gray}{0}
%\definecolor{background}{gray}{1}
%\definecolor{keyword}{rgb}{0.2,0.2,0.8}
%\definecolor{warning}{rgb}{0.8,0.2,0.2}
%\definecolor{shadow}{gray}{0.35}
%\definecolor{hide}{gray}{0.9}
%\definecolor{figure}{rgb}{1,0.7,0}
%\definecolor{title}{rgb}{25,240,250}
\definecolor{title}{rgb}{0.09,0.30,0.38}
\definecolor{frametitle}{rgb}{1,1,1}
\definecolor{normal}{rgb}{0,0,0}

%\usecolortheme[named=keyword]{structure}

\setbeamercolor{title}{fg=title}
\setbeamercolor{frametitle}{fg=frametitle}
\setbeamercolor{section in toc}{fg=foreground}
\setbeamercolor{normal text}{bg=brown!46,fg=normal}

\setbeamerfont{structure}{family=\fontspec{Georgia},series=\bfseries} 
\setbeamerfont{subtitle}{family=\fontspec{Helvetica},series=\bfseries} 
\begin{document}

\title{KRAD Training}
\subtitle{Maintenance Documents: Part 1}
\author[Leo]{Leo Przybylski}

\usebackgroundtemplate%
{%
    \includegraphics[width=\paperwidth,height=\paperheight]{../img/header.png}%
}

{
\usebackgroundtemplate{\includegraphics[width=\paperwidth]{../img/title.png}}%
\begin{frame}[plain]
  \titlepage
\end{frame}
}

\begin{frame}{Goals}
  \begin{itemize}
 \item Learn how defining a custom Maintainable for your maintenance documents can provide you with various hook points to implement more complex features.
 \item Implement a Maintainable for the Book maintenance document.
\end{itemize}
\end{frame}

\begin{frame}{Checkout “exercise-maint-maintainable” project}
  \begin{itemize}
    \item To ensure a clean and consistent environment for everyone, we will check out a project from Subversion as a starting point.  This will essentially be a completed copy of the previous exercises.
    \item In order to get the copy of the project that you will need,
      please check out the exercise-maint-maintainable project from
      the training Subversion repository.
      \end{itemize}
    \end{frame}
    
    \begin{frame}{Create and Configure the Maintainable}
      \begin{itemize}
      \item We will add some functionality to the Book maintenance document using a custom Maintainable class.  The lookup framework has the capability built into it to allow you to “copy” a business object (as opposed to “edit” an existing one).  This is a nice feature, but imagine if we wanted to make some improvements to the way this is handled by default for the Book maintenance document.
      \item A custom Maintainable can help us out here!  Using one of
        these we will implement it such that when a “copy” is
        performed on a Book, the ISBN is not carried over during the
        copy.  From a functional perspective, this makes sense because
        two books will never have the same ISBN.
      \end{itemize}     
    \end{frame}

    \begin{frame}{Create and Configure the Maintainable}
 To do this, follow these steps:
      \begin{itemize}
      \item Create a new package under src/main/java called \texttt{trnapp.bookstore.}
      \item Inside of this package, create a new class named
        BookMaintainable.  Have it extend from
        \texttt{org.kuali.rice.krad.maintenance.MaintainableImpl}.
      \end{itemize}     
    \end{frame}
 
 \begin{frame}[fragile]{Create and Configure Maintainable}
 In this new class implement the processAfterCopy method as follows:
 \begin{minted}[fontsize=\scriptsize]{java}
@Override
public void processAfterCopy(MaintenanceDocument document, Map<String, String[]> parameters) {
    super.processAfterCopy(document, parameters);
    Book book = ((Book)document.getNewMaintainableObject().getBusinessObject());
    book.setIsbn(null);
}
    \end{minted}
\end{frame}

    \begin{frame}{Create and Configure the Maintainable}
     \begin{itemize}
       \item Now, reconfigure your Book MaintenanceDocumentEntry in your BookMaintenanceDocument.xml file so that it references your new BookMaintainable class.
       \item Launch the web application and navigate to the Book lookup.
       \item Execute a search and find a Book with an ISBN on it.
         \item Click the “Copy” link on the far left-hand side.
         \item You should see something like the following (note how the ISBN was not copied to the new Book on the form):
      \end{itemize}     
    \end{frame}
           
\end{document}
